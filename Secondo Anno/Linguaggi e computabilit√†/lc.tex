\documentclass[12pt, a4paper, openany, oneside]{book}
\usepackage[italian]{babel}
\usepackage[T1]{fontenc}
\usepackage[utf8]{inputenc}
\usepackage{amsmath} 
\usepackage{xcolor}
\usepackage[margin=0.5in]{geometry}
%usepackage[latin1]{inputenc}
\begin{document}
\pagestyle{plain}
\author{DaveRhapsody}
\title{Linguaggi e Computabilità}
\date{2 Ottobre 2019}
\maketitle
\tableofcontents
\chapter*{L'esame}
Avremo due compitini, uno a novembre ed uno a Gennaio, in un anno sono 
disponibili 5 appelli, se uno è del terzo anno, può fare i compitini, basta che
ci sia spazio nelle aule, la precedenza va a coloro che sono del secondo anno.
\\ \\
Al secondo appello (Quello di Febbraio) puoi recuperare il voto negativo di uno
dei due compitini. Non presentarsi è esattamente come provarci e non passare, 
quindi rischiate, conviene.
\\ \\ 
L'orale va sostenuto nello stesso appello dello scritto, cioè io faccio lo 
scritto, lo passo, l'orale lo devo fare in quella sessione.
Per chi fa i compitini ed ha consegnato anche gli esercizi di lab. può fare un
orale prima del 5 Febbraio OPPURE si può fare assieme a coloro che hanno fatto
l'esame il 5.
\\ \\
Gli esercizi valgono dal momento che li invii fino a fine anno, quindi ha senso
farli subito tutti
\\
{\color{black} \rule{\linewidth}{0.3mm} }
\\
\chapter{Linguaggi formali}
Nascono per essere in grado di creare i linguaggi di programmazione, o meglio
servono per gestire i protocolli di comunicazione e la possibilità di comunicare
una determinata operazione al calcolatore.
\section{Backus-noun form}

\section{Model checking}
Usato per protocolli di comunicazione, per esempio per protocolli di pagamento,
in realtà di qualsiasi tipo, chiaramente per la sicurezza questo è l'ideale, 
perchè si descrive lo stato di sistema, e si specifica se ogni stato è sicuro
(Sicuro sia dal punto di vista dei risultati corretti che sicuri)
\\ \\
E' usato anche per il software, cioè in maniera automatica deduce in base alle
condizioni di ingresso, se son corrette. Ce la fa? Si per programmi piccini, 
ma alla fine, ma ingenerale, non esiste una tecnica che preso un software ti 
dimostra che esso sia corretto in ogni caso. Non esiste nessuna procedura generale,
se esistesse ci sarebbero contraddizioni logiche.
\paragraph{Cos'è una contraddizione logica?} 
E' un paradosso, ma a livello un po' più infame, pensate alla frase "Questa 
frase è vera", se ci scavate a fondo, dopo un po' diventa una contraddizione.
\section{Automi a stati finiti}
Sono insiemi di stati ai quali arrivan dall'esterno dei dati, ed a seconda dello
stato in cui si trovano, e del dato che arriva, allora potrebbero verificarsi le
famose "Transizioni" che consistono nel cambiare stato.
\\ \\ 
La memoria del Latch SR, ad esempio, funziona come un automa, nel senso, varia
a seconda dello stato interno, e del valore di ingresso.
\\ \\
\paragraph{Linguaggio Per} E' uno dei primi linguaggi di scripting, anche se ce 
n'era qualcun altro prima, e contiene istruzioni per gestire espressioni regolari
che possono essere applicate su testi lunghi per fare ricerche.\\ 
In pratia prendevano delle sequenze di DNA (tera di dati), e venivano analizzati
(con espressioni regolari) da questo linguaggio.
\chapter{Alfabeto}
E' un insieme finito e non vuoto di simboli, ad esempio:
{A, B, C, D, ... , Z} , {1, 2, 3, 4, ..., 9}. \\ 
Per gli alfabeti useremo lettere greche tipo: $\Sigma, \Lambda, \Gamma$, vediamo
alcune definizioni ora:
\paragraph{Stringa} La stringa è una sequenza di simboli, se è 
vuota si definisce vuota, può esistere. \\ Data una stringa w, si indica la sua
lunghezza con |w|. Per esempio: |acdas234| = 8, mentre se ho |$\epsilon$| = 0,
poichè si indica che una stringa è vuota dicendo che essa abbia solo una lettera
greca dentro
\paragraph{Concatenazione tra stringhe}
La concatenazione fa in modo che date due stringhe w, x l'ultimo carattere di x
sarà il successivo dell'ultimo di w. pertanto, w, x $\to$ w $\circ$ x = wx \\
Per esempio se ho una stringa vuota, e la concateno ad una stringa, otterrò la
stringa (3+0 fa 0, no? :)) $\to \epsilon \circ $ w = w
\subparagraph{NON Commutatività di una stringa}
Concatenare due stringhe non è sempre possibile, a meno che siano perfettamente
identiche
\paragraph{Potenze di un alfabeto} 
Prendiamo un alfabeto $\Sigma$ e per un k intero >= 0 $\Sigma ^{k} = \Sigma x, 
\Sigma x, \Sigma x, \Sigma x$, ottengo una permutazione di k volte $\Sigma$, tutte
appartenenti a $\Sigma ^{k}$ \\ \\
Come sarà la sua cardinalità? |$\Sigma$| = q $\to$  |$\Sigma ^{k}| - q^{k}$.
\\ \\ 
Per k = 1 avrei $\Sigma ^{1}$ w = qualsiasi elemento di $\Sigma$(un solo elemento) 
\\ \\
Se ho $\Sigma$ = {0, 1} \\
$\Sigma ^{2}$ = Tutte le permutazioni che posso fare con 0, 1 \textbf{i lunghezza 2}
(I valori di $\Sigma$)
\\ \\
Per definizione $\Sigma ^{0}$ = {$\epsilon$}
\end{document}