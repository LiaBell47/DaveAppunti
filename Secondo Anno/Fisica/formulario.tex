\documentclass[a4paper,12pt, oneside]{book}

%\usepackage{fullpage}
\usepackage[italian]{babel}
\usepackage[utf8]{inputenc}
\usepackage{amssymb}
\usepackage{amsthm}
\usepackage{graphics}
\usepackage{amsfonts}
\usepackage{amsmath}
\usepackage{amstext}
\usepackage{engrec}
\usepackage{rotating}
\usepackage{verbatim}
\usepackage[safe,extra]{tipa}
\usepackage{showkeys}
\usepackage{multirow}
\usepackage{hyperref}
\usepackage{microtype}
\usepackage{enumerate}
\usepackage{braket}
\usepackage{marginnote}
\usepackage{pgfplots}
\usepackage{cancel}
\usepackage{polynom}
\usepackage{booktabs}
\usepackage{enumitem}
\usepackage{framed}
\usepackage{pdfpages}
\usepackage{pgfplots}
\usepackage{fancyhdr}
 \usepackage[usenames,dvipsnames]{pstricks}
 \usepackage{epsfig}
 \usepackage{pst-grad} % For gradients
 \usepackage{pst-plot} % For axes
 \usepackage[space]{grffile} % For spaces in paths
 \usepackage{etoolbox} % For spaces in paths
 \makeatletter % For spaces in paths
 \patchcmd\Gread@eps{\@inputcheck#1 }{\@inputcheck"#1"\relax}{}{}
 \makeatother
\pagestyle{fancy}
\fancyhead[LE,RO]{\slshape \rightmark}
\fancyhead[LO,RE]{\slshape \leftmark}
\fancyfoot[C]{\thepage}


\title{Formulario di Fisica}
\author{UniShare\\\\Davide Cozzi\\\href{https://t.me/dlcgold}{@dlcgold}}
\date{}

\pgfplotsset{compat=1.13}
\begin{document}
\maketitle

\definecolor{shadecolor}{gray}{0.80}

\newtheorem{teorema}{Teorema}
\newtheorem{definizione}{Definizione}
\newtheorem{esempio}{Esempio}
\newtheorem{corollario}{Corollario}
\newtheorem{lemma}{Lemma}
\newtheorem{osservazione}{Osservazione}
\newtheorem{nota}{Nota}
\newtheorem{esercizio}{Esercizio}
\tableofcontents

\renewcommand{\chaptermark}[1]{%
	\markboth{\chaptername
		\ \thechapter.\ #1}{}}
\renewcommand{\sectionmark}[1]{\markright{\thesection.\ #1}}
\chapter{Introduzione}
\subsection{Trigonometria}
\begin{center}
	\begin{pspicture}(0,-2.1457813)(5.3,2.1457813)
		\psline[linecolor=black, linewidth=0.04](4.4,-1.6694922)(4.4,1.5305078)
		\psline[linecolor=black, linewidth=0.04](4.4,1.5305078)(0.8,-1.6694922)(4.4,-1.6694922)
		\rput[bl](4.4,-2.069492){$A$}
		\rput[bl](4.4,1.9305078){$C$}
		\rput[bl](0.0,-1.6694922){$B$}
		\rput[bl](2.0,-0.06949219){$a$}
		\rput[bl](4.7,-0.4494922){$b$}
		\rput[bl](2.4,-2.069492){$c$}
		\rput[bl](1.29,-1.5894922){$\beta$}
		\rput[bl](3.8,-1.3894922){$\alpha$}
		\rput[bl](4,0.7705078){$\gamma$}
	\end{pspicture}
	$$b= a \sin\beta$$
	$$c=a\sin \gamma$$
	$$b=a\cos \gamma$$
	$$c=a cos\beta$$
	$$c=b\tan \gamma$$
	$$b=c\tan \beta$$
	quindi su un piano inclinato:
	$$l=\frac{h}{\sin\theta}$$
\end{center}
inoltre:
$$\sin^2(\theta)+\cos^2(\theta)=1$$
$$\sin(2\theta)=2\sin(\theta)\cos(\theta)$$
$$\cos(2\theta)=\cos^2(\theta)-\sin^2(\theta)$$
\section{vettori}
prodotto scalare tra vettori: $\overline{x}\cdot \overline{y}=||x||\cdot ||y||\cdot \cos\theta$\\
prodototto  vettoriale tra vettori: $\overline{x}\times \overline{y}=||x||\cdot ||y||\cdot \sin\theta$
\section{Costanti}
\begin{itemize}
	\item \textbf{accelerazione di gravità:} $g=9,81\, \frac{m}{s^2}$
	\item \textbf{costante gravitazionale:} $6,67 \times 10^{-11}\, \frac{m^3}{s^2kg}$
	\item \textbf{raggio terra:} $R_L=6,37\times 10^{6}\, m$
	\item \textbf{massa terra:} $M_T=5,96\times 10^{24}\, kg$
	\item \textbf{massa sole:} $M_S =1,99\times 10^{30}\, kg$
	\item \textbf{massa luna:} $M_L=7.36\times 10^{22}\, kg$
\end{itemize}
\chapter{Meccanica}
\section{Cinematica}
\subsubsection{Moto rettilineo}
\begin{itemize}
	\item \textbf{velocità media: }$v_m=\frac{\Delta \vec{x}}{\Delta t}=\frac{x_2-x_1}{t_2-t_1}=\frac{\vec{v_2}-\vec{v_1}}{2}$
	\item  \textbf{velocità istantanea: }$v(t)=\frac{d\vec{x}(t)}{dt}$
	\item \textbf{equazione del moto rettilineo uniforme: }$x(t)=x_0+v(t-t_0)$
	\item \textbf{accelerazione media:} $a_m=\frac{\vec{v_2}-\vec{v_1}}{t_2-t_1}=\frac{\Delta v}{\Delta t}$
	\item \textbf{velocità moto uniformemente accelerato:} $v(t)=v_0+at$
	\item \textbf{equazione del moto rettilineo uniformemente accelerato:} $$x(t)=x_0+v_0t+\frac{a}{2}t^2$$
	\item \textbf{velocità finale moto uniformemente accelerato:}
	      $$v_{fin}^2=v_0^2+2a\Delta x$$
\end{itemize}
\newpage
\subsubsection{Moto verticale}
\begin{itemize}
	\item \textbf{punto ad altezza h lasciato cadere:}
	      $$\vec{x}(t)=h-\frac{1}{2} g t^2$$
	      $$\vec{v}(t)=-gt$$
	      $$t_{caduta}=\sqrt{\frac{2 h}{g}}$$
	      $$\vec{v}_{suolo}=-\sqrt{2 g h}$$
	\item \textbf{punto ad altezza h spinto in basso con una certa velocità verso il basso:}
	      $$\vec{x}(t)=h-\vec{v}_1t-\frac{1}{2} g t^2$$
	      $$\vec{v}(t)=-\vec{v}_1-gt$$
	      $$t_{caduta}=-\frac{\vec{v}_1}{g}+\frac{1}{g}\sqrt{\vec{v}_1^2+2gh}$$
	      $$v_{suolo}=-\sqrt{\vec{v_1}^2+2gh}$$
	\item \textbf{punto ad altezza 0 spinto in alto con una certa velocità:}
	      $$\vec{x}(t)=\vec{v_2}t-\frac{1}{2} g t^2$$
	      $$\vec{v}(t)=\vec{v_2}-gt$$
	      con $v=0$ si ha l'altezza massima:
	      $$t_{x_{max}}=\frac{\vec{v_2}}{g}$$
	      e quindi:
	      $$x(t_{max})=\frac{1}{2}\frac{\vec{v_2}^2}{g}$$
	      $$t_{caduta}=\frac{\vec{v_2}}{g}$$
	      $$t_{tot}=t_{max}+t_c=\frac{2\vec{v_2}}{g}$$
\end{itemize}
\begin{comment}
\subsubsection{Moto nel Piano}\textbf{da sistemare}
\begin{itemize}
	\item \textbf{modulo della velocità in componenti cartesiane}: $$v=|\vec{v}|=\sqrt{v_x^2+v_y^2}$$
	\item \textbf{modulo della velocità in componenti cartesiane}:
	      $$v=|\vec{v}|=\sqrt{v_r^2+v_q^2}$$
	\item \textbf{accelerazione nel piano: }$\vec{a}=\vec{a}_T+\vec{a}_n$
\end{itemize}
\end{comment}
\subsubsection{Moto Circolare}
\begin{itemize}
	\item \textbf{arco:} $l_a=\frac{\Delta s}{R}$
	\item \textbf{angolo:} $\theta=\frac{l_a}{R}$
	\item \textbf{velocità angolare media nel moto uniforme:} $\omega_m=\frac{\Delta\theta}{\Delta t}$
	\item \textbf{velocità angolare istantanea nel moto uniforme:} $\omega=\frac{v}{R}$
	\item \textbf{accelerazione centripeta (quella tangenziale è nulla) nel moto uniforme:}
	      $$a=\frac{v^2}{R}=\omega^2R=\omega v$$
	\item \textbf{equazioni del moto uniforme:}
	      $$s(t)=s_0+vt$$
	      $$\theta(t)=\theta_0+\omega t$$
	\item \textbf{periodo:}
	      $$T=\frac{2\pi R}{v}=\frac{2\pi R}{\omega R}=\frac{2\pi}{\omega}$$
	\item \textbf{accelerazione nel caso di moto non uniforme:}
	      $$\vec{a}=\vec{a}_T+\vec{a}_N$$
	      $$\alpha_{media}=\frac{\Delta\omega}{\Delta t}$$
	      $$\alpha_{istantanea}=\frac{1}{R}a_T$$
	      $$a_N=\omega^2 R$$
	      $$a_T=\alpha R$$
	\item \textbf{equazioni del moto circolare non uniforme:}
	      $$\omega(t)=\omega_0+\alpha t$$
	      $$\theta(t)=\theta_0+\omega_0t+\frac{1}{2}\alpha t^2$$
	      $$a_N=\omega^2 R=(\omega_0+\alpha t)^2 R$$
	      $$|\vec{v}|=\omega R$$
\end{itemize}
\subsubsection{Moto Parabolico}
\begin{itemize}
	\item \textbf{moto parabolico da terra, con angolo e velocità iniziale:}
	      $$
		      \begin{cases}
			      v_x=v_0cos\theta_0 \\
			      v_y=v_0sin\theta_0-gt
		      \end{cases}
	      $$
	      $$\begin{cases}
			      x(t)=(v_0cos\theta_0)t \\
			      y(t)=(v_0sin\theta_0)t-\frac{1}{2}gt^2
		      \end{cases}$$
	      $$t=\frac{x}{v_0cos\theta_0}$$
	      $$y(x)=(tan\theta_0)x-\frac{g}{2v_0^2cos^2\theta_0}x^2\mbox{ (traiettoria)}$$
	      $$x_G=\frac{v_0^2}{g}sin(2\theta_0)\mbox{ (gittata, y(x)=0)}$$
	      $$x_G=\frac{v_0^2}{g}\mbox{ (gitatta massima)}$$
	      $$x_M=\frac{1}{2}\frac{v_0^2}{g}sin(2\theta_0)\mbox{ (altezza massima)}$$
	      $$y_M=\frac{v_0^2}{2g}sin^2\theta_0 \mbox{ (altezza massima lungo la traiettoria)}$$
	      $$Y_{M_{max}}=\frac{v_0^2}{2g}\mbox{ (altezza massima, la verticale)}$$
	      $$t_{volo}=\frac{2v_0}{g}sin\theta_0$$
	      $$t_{{volo}_{max}}=\frac{2v_0}{g}$$
	      $$\begin{cases}
			      v_x(t_G)=v_x(t_0)=v_0cos\theta_0 \\
			      v_y(t_G)=-v_y(t_0)=-v_0sin\theta_0
		      \end{cases}\mbox{ (velocità finali)}$$
	\item \textbf{moto parabolico da altezza h:}
	      $$\begin{cases}
			      x(t)=v_0t \\
			      y(t)=h-\frac{1}{2}gt^2
		      \end{cases}$$
	      $$\begin{cases}
			      v_x(t)=v_0 \\
			      v_y(t)=-gt
		      \end{cases}$$
	      $$t_{volo}=\frac{x}{v_0}$$
	      $$y(x)=h-\frac{g}{2v_0^2}x^2\mbox{ (traiettoria)}$$
	      $$t_{caduta}=\sqrt{\frac{2h}{g}}$$
	      $$x(t_c)=x_G=v_0t_c=v_0\sqrt{\frac{2h}{g}}\mbox{ (gittata)}$$
	      $$\begin{cases}v_x(t_c)=v_0 \\
			      v_y(t_c)=-\sqrt{2gh}\end{cases} \mbox{( velocità finali)}$$
	      $$v_{caduta}=\sqrt{v_0^2+2gh}$$
\end{itemize}
\section{Dinamica}
\begin{itemize}
	\item \textbf{seconda legge della dinamica:} $\vec{F}=m\vec{a}$
	\item \textbf{forza elastica:}
	      $$\vec{F}_e=-k\Delta \vec{x}$$
	      $$\vec{a}=\frac{-k(x-x_0)}{m}$$
	\item \textbf{forza peso:}
	      $\vec{F}_p=mg$
	\item \textbf{forza d'attrito:}
	      $$\vec{f}_{AD}=-\mu_DN$$
	      $$\vec{f}_{AS}=-\mu_SN$$
	\item \textbf{lunghezza piano inclinato:} $$L=\frac{h}{sin\theta}$$
\end{itemize}
\subsubsection{Lavoro e Energia}
\begin{itemize}
	\item \textbf{lavoro:}
	      $$L=\vec{F}_x\vec{\Delta x}$$
	      $$L=|\vec{F}|\,|\vec{\Delta x}|cos\theta=\vec{F}\vec{s}$$
	\item \textbf{energia cinetica:}
	      $E_k=\frac{1}{2}mv_f^2-\frac{1}{2}mv_0^2$
	\item \textbf{energia potenziale}
	      $E_P=mgz_B-mgz_A$
	\item \textbf{lavoro della forza elastica:}
	      $E_{Pe}=\frac{1}{2}kx^2$
	\item \textbf{lavoro della forza d'attrito:}
	      $W_{AD}=-\mu_DNl_{AB}$
	\item \textbf{conservazione dell'energia meccanica con forze conservative:}
	      $$E_{KB}+E_{PB}=E_{KA}+E_{PA}$$
	\item \textbf{conservazione dell'energia meccanica con forze conservative:}
	      $$E_{KB}+E_{PB}-E_{KA}+E_{PA}=E_{MB}-E_{MA}=\Delta E_M$$
	      $$W=W_{cons}+W_{non-cons}$$
	      $$W_{non-cons}=\Delta E_M$$
	\item \textbf{energia meccanica nel caso di presenza di forze d'attrito:}
	      $$\Delta E_M=-\mu_DNl_{AB}$$
\end{itemize}
\subsubsection{Piano inclinato}
\begin{center}
	\begin{pspicture}(0,-2.1787758)(5.26,2.1787758)
		\psline[linecolor=black, linewidth=0.04](0.82,1.0412241)(1.62,1.8412242)(2.42,1.0412241)(1.62,0.24122417)(0.82,1.0412241)
		\psline[linecolor=black, linewidth=0.04](1.22,0.64122415)(4.02,-2.1587758)(0.02,-2.1587758)(0.02,1.8412242)(1.22,0.64122415)
		\psline[linecolor=black, linewidth=0.04, linestyle=dashed, dash=0.17638889cm 0.10583334cm, arrowsize=0.05291667cm 2.0,arrowlength=1.4,arrowinset=0.0]{->}(2.02,0.64122415)(3.22,-0.55877584)
		\psline[linecolor=black, linewidth=0.04, linestyle=dashed, dash=0.17638889cm 0.10583334cm, arrowsize=0.05291667cm 2.0,arrowlength=1.4,arrowinset=0.0]{->}(2.02,1.4412242)(2.82,2.241224)
		\rput[bl](2.82,-1.7587758){$\theta$}
		\rput[bl](2.82,0.24122417){$mg\sin\theta$}
		\rput[bl](2.82,1.6412242){$mg\cos\theta$}
	\end{pspicture}\\
	\textbf{forza normale:} $N=mg\cos\theta$\\
	\textbf{lavoro attrito:} $W_{AD}=\mu_Dmg\cos\theta l_{AB}$
\end{center}
\section{Gravitazione}
\begin{itemize}
	\item terza legge di Keplero:
	      $$T^2=k_Sa^3$$
	      con
	      $$r_1+r_2=2a$$
	\item legge di gravitazione universale:
	      $$F=-G\frac{m_1m_2}{r^2}\vec{u}_r$$
	      $$G=6,67 \times 10^{−11} \frac{N m^2}{kg^2}$$
	      $$g=\frac{Fm_T}{r_T^2}$$
	      $$G=6,67\times10^{-11}\frac{Nm^2}{kg^2}6,67\times10^{-11}\frac{Nm^3}{s^2kg}$$
	      $$g=G\frac{M_t}{r_T^2}$$
	\item campo gravitazionale:
	      $$\vec{\eta}(\vec{r})=\left(-G\frac{M}{r^2}\vec{u}_r\right)$$
	      $$\vec{\eta}(P)=\sum \vec{\eta}_i=-g\sum \frac{M_i}{r_i^2}\vec{u}_i$$
	\item energia potenziale gravitazionale:
	      $$E_P=-G\frac{Mm}{r}$$
	\item velocità di fuga:
	      $$\frac{1}{2}mv_f^2-G\frac{Mm}{r}=0$$
	      $$\downarrow$$
	      $$v_f=\sqrt{\frac{2GM}{r}}$$
	\item velocità orbitale:

	      $$F=m\cdot \frac{v^2}{r}$$
	      $$\downarrow$$
	      $$\frac{Mm}{r^2}=m\cdot \frac{v^2}{r}$$
	      $$v=\sqrt{\frac{GM}{r}}$$
	\item teorema del guscio:
	      $$\rho=\frac{M_T}{V_T}=\frac{M_T}{\frac{4}{3}\pi r^3}$$
	      $$F=G\frac{m_Tm}{r_T^2}=G\frac{\rho\frac{4}{3}\pi r^3 m}{r^2}=G\frac{4}{3}\rho m r \mbox{ che con } k=G\frac{4}{3}\rho m \to F=-kr \mbox{ negativo attrazione}$$
	\item energia:
	      $$E_P=G\frac{Mm}{r}$$
	      $$E_K=\frac{1}{2}mv^2=\frac{1}{2}m\omega^2r^2$$
	      $$G\frac{Mm}{r^2}=ma=m\frac{v^2}{r}\to\omega^2r^2=G\frac{m}{r}$$
	      $$E_K=G\frac{Mm}{2r}$$
	      $$E_M=E_K+E_P=G\frac{Mm}{2r}-G\frac{Mm}{r}=-G\frac{Mm}{2r}$$
\end{itemize}
\newpage
\section{Moto Armonico}
\begin{itemize}
	\item equazioni del moto:
	      $$x(t)=Acos(\omega t+\varphi)$$
	      $$v(t)=-Asin(\omega t+\varphi)$$
	      $$a(t)=-\omega^2Acos(\omega t+\varphi)$$
	      $$T=\frac{2\pi}{\omega}$$
	\item dinamica:
	      $$F=2\pi\sqrt{\frac{m}{k}}$$
	      $$E_K=\frac{1}{2}m\omega^2A^2sin^2(\omega T)$$
	      $$E_U=\frac{1}{2}kA^2cos^2(\omega T)$$
	      $$E_M=E_K+E_U=\frac{1}{2}kA^2$$
	\item pendolo:
	      $$F_p=-mg\sin\theta$$
	      $$F=ma=-mg\sin\theta $$
	      $$\frac{d^2x}{dt^2}=-g\sin\theta$$
	      $$x=-\frac{g}{L}\sin\theta$$
	      $$\theta(t)=\theta_{max}\cos(\omega t+\phi)$$
	      $$ \omega=\sqrt{\frac{g}{L}}$$
	      $$t=\frac{2\pi}{\omega}=2\pi\sqrt{\frac{L}{g}}$$
\end{itemize}
\newpage
\section{Fluidodinamica}
\begin{itemize}
	\item densità:
	      $$\rho=\frac{m}{V}$$
	\item pressione:
	      $$p=\frac{F}{S}$$
	      $$dW_P=df\,dh=pdS\,dh=p\,dV$$
	      $$dF_{peso}=g\,dm=g\rho dV=g\rho \,dS\,dh$$
	      $$dF_{pressione}=-dp\,dS=[p(h)-p(h+dh)]dS$$
	\item equilibrio:
	      $$dF_{peso}+dF_{pressione}=0$$
	      $$\downarrow$$
	      $$g\rho \,dS\,dh-dp\,dS=0$$
	      $$\downarrow$$
	      $$g\rho \,dh-dp=0$$
	      $$\downarrow$$
	      $$\frac{dp}{dh}=g\rho$$
	\item peso apparente:
	      $$F_p=(\rho_{corpo}-\rho_{fluido})V_{corpo}g$$
	\item legge di Stevino:
	      $$p(h)=p_0+g\rho h$$
	\item principio di Archimede:
	      $$F_A=g\rho V$$
	\item portata:
	      $$q=vS=costante$$
	\item equazione di continuità:
	      $$v_1S_1=v_2S_2$$
	\item teorema di Bernoulli:
	      $$p+\frac{1}{2}\rho v^2+g\rho z =costante$$
	\item teorema di Torricelli
	      $$\Delta z= h, v_A=0, S_A>>S_a, P_A=p_0\to v_a=\sqrt{2gh}$$
\end{itemize}
\section{Termodinamica}
\begin{itemize}
	\item scale termiche:
	      $$t(^{\circ}C)=T(K)-273,15$$
	      $$t(^{\circ}R)=\frac{9}{5} T(K)$$
	      $$t(^{\circ}F)=\frac{9}{5} T(K)-459,67$$
	      $$t(^{\circ}F)=\frac{9}{5} T(^{\circ}C)+32$$
	      $$t(^{\circ}C)=\frac{5}{9} [T(^{\circ}F)-32]$$
	\item legge isoterma di Boyle per i gas perfetti:
	      $$T=costante\to pV=costante\to p_1V_1=p_2V_2$$
	\item legge isobara di Volta-Gay Lussac, $\alpha$ coefficiente di dilatazione termica, dipendente dal gas:
	      $$p=costante\to V=V_0(1+\alpha t)$$
	\item legge isocora di Volta-Gay Lussac, $\beta$ costante indipendente dal gas, $t$ temperatura in celsius:
	      $$V=costante\to p=p_0(1+\beta t$$
	\item proprietà dei gas perfetti:
	      $$\alpha=\beta=\frac{1}{273,15} {\,}^\circ C^{-1}$$
	      $$V=V_0\alpha\left(\frac{1}{\alpha}+t\right)=V_0\alpha T$$
	      $$p=p_0\alpha\left(\frac{1}{\alpha}+t\right)=p_0\alpha T$$
	\item moli:
	      $$N_{molecole}=\frac{M_{gas}}{m_{molare}}$$
	      $$m_{molecola}=M_{molecolare}m_{atomica}=M_{molecolare}1,6604\times 10^{-27}$$
	      $$N_{Avogadro}=6,0221\times 10^{-23}\left[\frac{molecole}{moli}\right]$$
	      $$volume_{molare}=0,022314\,m^3$$
	      $$N_{paricelle}=n_{moli}N_{avogadro}$$
	\item Legge dei gas perfetti:
	      $$pV=nRT=Nk_BT$$
	      costante del gas perfetto:
	      $$R=8,314\,\frac{J}{mol}K=8314\,\frac{J}{kmol}K$$
	      costante di Boltzmann:
	      $$k_B=\frac{R}{N_A}=\frac{8,314}{6,0221\times 10^{23}}=1,3807\times 10^{-23}\frac{J}{K}$$
	      $$\frac{p}{\rho}=\frac{RT}{A}$$
	\item energia interna di tutte le trasformazioni:
	      $$\Delta U=\frac{3}{2}nR\Delta T$$
	\item lavoro e calore delle trasformazioni:
	      \begin{itemize}
		      \item isoterma: $W=Q=nRT\,ln\left(\frac{V_B}{V_A}\right)$
		      \item isobara: $W=W=p(V_B-V_A)=p\Delta V$ e
		            $Q=c_{specifico-molare-a-p-costante}n\Delta T$
		      \item isocora: $W=0$ e $Q=c_{specifico-molare-a-V-costante}n\Delta T=\frac{3}{2}R\Delta T$
		      \item adiabatica $\Delta U=-W$ e $Q=0$, $$\frac{p_aV_a}{T_a}=\frac{p_bV_b}{T_b}=\frac{p_cV_c}{T_c}=nR$$
	      \end{itemize}
	\item esperimento di Joule sull'aumento della temperatura:
	      $$Q=\Delta U=-W$$
	      lavoro positivo se ceduto all'esterno
	\item teoria cinetica, $E_k$ energia cinetica media:
	      $$pV=\frac{3}{2}N_{avogadro}E_k$$
	      $$E_k=\frac{3}{2}kT$$
	\item energia interna, $n$ numero molecole:
	      $$U=nE_k=\frac{3}{2}nRT$$
	\item velocità quadratica media.
	      $$V_{qm}=\sqrt{\frac{3RT}{M}}$$
	\item primo principio della termodinamica:
	      $$Q-W=\Delta U\to Q=\Delta U+W$$
	\item trasformazione ciclica:
	      $$\Delta U=0\to Q=W$$
	      $Q>0$ assorbe calore e fornisce lavoro $W>0$
	\item trasformazione adiabatica:
	      $$Q=0$$
	\item calore per far cambiare temperatura:
	      $$Q=mc_{specifico}\Delta T$$
	\item capacità termica:
	      $$C=mc$$
	\item calore specifico molare:
	      $$c=\frac{1}{n}\frac{dQ}{dT}$$
	\item calore per un cambio di fase, $\lambda$ calore latente:
	      $$Q=m\lambda$$
	\item $1 cal=4,1J$
	\item primo principio per i gas ideali se $c_v$ è costante:
	      $$dQ=nc_vdT+dW\longrightarrow Q=nc_v\Delta T+W$$
	      $$\Delta U(T)=nc_v\Delta T$$
\end{itemize}
\section{Elettrostatica}
\begin{itemize}
	\item particelle elementari:
	      \begin{itemize}
		      \item neutrone: $q_n=0$ $m_n=1,67\times 10^{-27}$
		      \item protone: $q_p1,621\times 10^{-19}$ w $m_p=1,67\times 10^{-27}$
		      \item elettrone: $q_e=e-1,6022\times 10^{-19}$ $m_e=9,11\times 10^{-31}$
	      \end{itemize}
	\item legge di Coulomb:
	      $$F=k\frac{q_1q_2}{r^2}$$
	\item conseguenze forza di Coulomb:
	      $$\frac{q_1}{q_2}=\frac{R_1}{R_2}$$
	      $$\frac{F_1}{F_2}=\frac{q_1}{q_2}$$
	\item costante k e costante dielettrica nel vuoto:
	      $$k=8,9875\times 10^9\sim 9\times 10^9\,\frac{Nm^2}{C^2}$$
	      $$k=\frac{1}{4\,\pi\,\varepsilon_0}$$
	      $$\varepsilon_0=\frac{1}{4\,\pi\,k}=8,8542\times 10^{-12}\frac{C^2}{Nm^2}$$
	\item seconda forma forza di Coulomb:
	      $$F=\frac{1}{4\,\pi\,\varepsilon_0}\frac{q_1q_2}{r^2}$$
	\item principio di sovrapposizione:
	      $$F=\sum_i F_i=\sum_i \frac{1}{4\,\pi\,\varepsilon_0}\frac{q_iq_0}{r_i^2}=q_0\sum_i\frac{1}{4\,\pi\,\varepsilon_0}\frac{q_i}{r_i^2}$$
	\item campo elettrostatico:
	      $$E=\frac{F}{q_0}\left[\frac{N}{C}\right]$$
	      $$E=\frac{1}{4\,\pi\,\varepsilon_0}\frac{q_1}{r_1^2}$$
	\item flusso del campo elettrico, se positivo uscente:
	      $$d\Phi(E)=E\cdot u_nd\Sigma=E\cos \theta d\Sigma=E_nd\Sigma$$
	\item angolo solido:
	      $$\Omega=\frac{\Sigma_0}{r^2}$$
	\item flusso di una carica attraverso una superficie finita:
	      $$d\Phi(E)=\frac{q}{4\,\pi\,\varepsilon_0}\frac{u_r\cdot u_nd\Sigma}{r^2}=\frac{q}{4\,\pi\,\varepsilon_0}\frac{d\Sigma\cos \theta}{r^2}=\frac{q}{4\,\pi\,\varepsilon_0}\frac{d\Sigma_0}{r^2}$$
	      $$d\Phi(E)=\frac{q}{4\pi\varepsilon_0}d\Omega$$
	      $$\Phi(E)=\int_\Sigma E\cdot u_nd\Sigma=\frac{q}{4\pi\varepsilon_0}\int d\Omega=\frac{q}{4\pi\varepsilon_0}\Omega$$
	\item carica interna alla superficie chiusa, Legge di Gauss:
	      $$\Phi(E)=\frac{q}{4\pi\varepsilon_0}4\pi=\frac{q}{\varepsilon_0}$$
	\item carica esterna alla superficie chiusa:
	      $$\Phi(E)=0$$
	\item principio di sovrapposizioni per cariche interne alla superficie:
	      $$\Phi(E)=\frac{1}{\varepsilon_0}(\Sigma_iq_i)_{int}$$
	\item flusso per carica puntiforme:
	      $$\begin{cases}
			      \Phi=ES_{sfera}=4\pi r^2 E \\
			      \Phi=\frac{q_0}{\varepsilon_0}
		      \end{cases}
	      $$
	      $$E=\frac{q_0}{4\pi r^2 \varepsilon_0}$$
	\item flusso su filo infinito lungo $h$:
	      $$\sum \Phi=\sum E\Delta S_i=E2\pi Rh$$
	      $$\Phi=2\Phi_{base}+\phi_{SL}=E2\pi Rh$$
	      $$E=\frac{1}{2\pi\varepsilon}\frac{\lambda}{r}$$
	      $$\lambda=\frac{q}{h}$$
	\item distribuzione planare:
	      $$\varepsilon_0\oint EdA=q_{enc}$$
	      $$\varepsilon_0(EA+EA)=\sigma A$$
	      $$E=\frac{\sigma}{2\sigma_0}$$
	\item lavoro campo elettrico:
	      $$dW=Fds=q_0Eds=q_0E\cos \theta ds$$
	      campo e spostamento paralleli:
	      $$dW=q_0E\,ds$$
	\item tensione elettrica:
	      $$T=\frac{W}{q_0}$$
	\item lavoro percorso chiuso, $\xi$ circuitazione:
	      $$W=q_0\xi=0$$
	\item potenziale elettrostatico:
	      $$V_A-V_B=\int_A^B Eds$$
	\item differenza di potenziale:
	      $$W_{AB}=q_0(V_A-V_B)=q_0\Delta V$$
	\item energia potenziale:
	      $$U=q_0\Delta V$$
	\item per una carica puntiforme:
	      $$dW=q_0Eds=\frac{q_0q}{4\pi\varepsilon_0}\frac{u\,ds}{r^2}=\frac{q_0q}{4\pi\varepsilon_0}\frac{dr}{r^2}$$
	      $$Eds=\frac{q}{4\pi\varepsilon_0}\frac{dr}{r^2}$$
	\item lavoro con spostamento tra A e B:
	      $$\int_A^B Eds=\frac{q}{4\pi\varepsilon_0}\int_{r_A}^{r_B}\frac{dr}{r^2}=\frac{q}{4\pi\varepsilon_0r_A}-\frac{q}{4\pi\varepsilon_0r_B}$$
	      $$W=q_0\int_A^B Eds=\frac{q_0q}{4\pi\varepsilon_0r_A}-\frac{q_0q}{4\pi\varepsilon_0r_B}$$
	      $$V_A-V_B=\frac{q}{4\pi\varepsilon_0r_A}-\frac{q}{4\pi\varepsilon_0r_B}$$
	      $$U_e(A)-U_e(B)=\frac{q_0q}{4\pi\varepsilon_0r_A}-\frac{q_0q}{4\pi\varepsilon_0r_B}$$
	      $$V(r)=\frac{q}{4\pi\varepsilon_0r}+A$$
	      $$U_e(r)=\frac{q_0q}{4\pi\varepsilon_0r}+B$$
	\item energia potenziale elettrostatica:
	      $$U_e=\frac{q_0q}{4\pi\varepsilon_0r}$$
	\item caso particolare:
	      $$V(\infty)=U(\infty)=0$$
	\item energia cinetica, $d$ distanza percorsa:
	      $$E_K=\frac{1}{2}mv^2=q\Delta V=q_0Ed$$
	\item superficie equipotenziale:
	      $$V(r)=\frac{q}{4\pi\varepsilon_0r}=costante\longrightarrow r=costante$$
	\item conduttori, superficie equipotenziale:
	      $$E=0\,\,\,all'interno$$
	      $$V(P_1)-V(P_2)=0\longrightarrow V(P_1)=V(P_2)=V_0$$
	\item densità superficiale:
	      $$\sigma=\frac{dq}{d\Sigma}$$
	\item teorema di Coulomb:
	      $$E=\frac{\sigma}{\varepsilon_0}u_n$$
	\item capacità di un conduttore:
	      $$C=\frac{q}{V_1-V_2}=\frac{4\pi\varepsilon_0R_1R_2}{R_2-R_1}$$
	\item capacità conduttore isolato:
	      $$C=\frac{q}{V}$$
	\item capacità condensatore, tr due conduttori isolati:
	      $$C=\frac{q}{V_1-V_2}$$
	      $$q=C(V_1-V_2)$$
	      $$V_1-V_2=\frac{q}{C}$$
	\item collegamenti di condensatori:
	      \begin{itemize}
		      \item parallelo:\\
		            carica sul conduttore superiore:
		            $$q=q_1+q_2=(C_1+C_2)V$$
		            carica sul conduttore inferiore:
		            $$-q=-(q_1+q_2)V$$
		            capacità equivalente del sistema:
		            $$C_{eq}=\frac{q}{V}=C_1+C_2$$
		            con $n$ condensatori:
		            $$C_{eq}=C_1+C_2+\cdots+C_n$$
		      \item serie:
		            $$V_C-V_A=\frac{q}{C_1}$$
		            $$V_B-V_A=\frac{q}{C_2}$$
		            $$V=V_C-V_A=\frac{q}{C_1}+\frac{q}{C_2}=q\left(\frac{1}{C_1}\frac{1}{C_2}\right)=\frac{q}{C_{eq}}$$
		            $$\frac{1}{C_{eq}}=\frac{1}{C_1}\frac{1}{C_2}\longrightarrow C_{eq}=\frac{C_1C_2}{C_1+C_2}$$
		            per $n$ condensatori:
		            $$\frac{1}{C_{eq}}=\frac{1}{C_1}\frac{1}{C_2}+\cdots+\frac{1}{C_n}$$
	      \end{itemize}
	\item velocità media elettroni:
	      $$v_m=\frac{1}{N}\sum v_i=0$$
	\item intensità di corrente:
	      $$i=\lim_{\Delta t\to 0}\frac{\Delta q}{\Delta t}=\frac{dq}{dt}$$
	\item densità di corrente:
	      $$j=n_+ev_d$$
	      $$di=ju_nd\Sigma$$
	\item rapporto corrente e flusso:
	      $$i=\Phi_\Sigma(j)$$
	      se superficie ortogonale e densità di corrente:
	      $$i=j\Sigma$$
	\item cammino medio tra urti:
	      $$\tau=\frac{l}{v}$$
	\item accelerazione in caso di campo elettrico:
	      $$a=\frac{F}{m}=-e\frac{E}{m}$$
	\item velocità di deriva:
	      $$v_{i+1}=v_i-\frac{eE}{m}\tau$$
	      $$v_d=\frac{1}{N}\sum_i v_{i+1}=-\frac{e\tau}{m}E$$
	\item seconda forma della densità di carica:
	      $$j=-nev_d=\frac{ne^2\tau}{m}E$$
	\item conduttività:
	      $$\sigma=\frac{ne^2\tau}{m}$$
	\item Legge di Ohm della conduzione elettrica:
	      $$j=\sigma E\to E=\rho j$$
	\item resistività del conduttore:
	      $$\rho =\frac{1}{\sigma}$$
	\item conduttore metallico cilindrico:
	      $$j=\frac{1}{\rho}E=\sigma E$$
	      $$i=j\Sigma=\frac{\Sigma}{\rho}E$$
	      $$V=\int_A^BEds=Eh$$
	      $$V=\frac{\rho h}{\Sigma}i$$
	\item resistenza  del conduttore:
	      $$R=\rho\frac{h}{\Sigma}$$
	\item legge di Ohm per i conduttori metallici:
	      $$V=Ri$$
	\item conduttanza:
	      $$G=\frac{1}{R}=\frac{\Sigma}{\rho h}=\frac{\sigma\Sigma}{h}$$
	\item collegamenti tra resistori:
	      \begin{itemize}
		      \item serie, corrente che attraversa i resistori costante:\\
		            differenza di potenziale ai campi delle resistenze:
		            $$V_{tot}=(R_1+R_2)i=R_{eq}i$$

		      \item parallelo differenza di potenziale agli estremi uguale :
		            condizione di stazionarietà:
		            $$i_f=i_1+i_2+\cdots i_n$$
		            calcolo corrente:
		            $$i=\frac{V}{R_1}+\frac{V}{R_2}=V\left(\frac{1}{R_1}+\frac{1}{R_2}\right)=\frac{V}{R_{eq}}$$
		            resistenza equivalente:
		            $$\frac{1}{R_{eq}}=\frac{1}{R_1}+\frac{1}{R_2}+\cdots+\frac{1}{R_n}$$
	      \end{itemize}
	\item leggi di Kirchhoff:
	      \begin{itemize}
		      \item prima legge (dei nodi):
		            $$\sum_ki_k=0$$
		      \item seconda legge (delle maglie). $\xi$ forza elettromotrice:
		            $$\sum_kR_ki_k=\sum_k\xi_k$$
	      \end{itemize}
	\item rapporto tra numero nodi N e numero rami R:
	      $$M=R-(N-1)$$
\end{itemize}
\section{Accenni di Magnetismo}
\begin{itemize}
	\item flusso:
	      $$\oint Bu_nd\Sigma=0$$
	\item forza di Lorentz:
	      $$F=qv\times B=qv\sin\theta B$$
	      $$F=q(E+v\times B)$$
	\item il campo magnetico si misura in Tesla T
	\item campo magnetico attraverso cilindro lungo $l$ e di base $A$:
	      $$i=nq_ev_iA$$
	      $$F=-q_ev_d\times B$$
	      $$N=nLA$$
	      $$F=-Nq_ev_d\times B=-nLAq_e(v_d\times B)=iL\times B$$
	\item prima legge elementare di Laplace, $k_m$ costante nel vuoto, $\mu_0$ permeabilità magnetica nel vuoto:
	      $$dB=k_mi\frac{ds\times u_r}{r^2}=k_m\frac{ids}{r^2}u_t\times u_r$$
	      $$k_m=10^{-7}\frac{Tm}{A}=10^{-7}\frac{H}{m}$$
	      $$k_m=\frac{\mu_0}{4\pi}$$
	      $$\mu_0=1,26\times 10^{-6}\frac{H}{m}$$
	\item seconda forma della prima legge elementare di laplace:
	      $$dB=\frac{\mu_0i}{4\pi}\frac{ds\times u_r}{r^2}=\frac{\mu_0}{4\pi}\frac{ids}{r^2}u_t\times u_r$$
	\item legge di Ampère-Laplace, campo magnetico in citcuito chiuso:
	      $$B=\frac{\mu_0i}{4\pi}\oint\frac{ds\times u_r}{r^2}$$
	\item filo indefinito rettilineo:
	      $$dB=\frac{\mu_0i}{4\pi}\frac{ds\sin\theta}{r^2}$$
	      a metà filo:
	      $$B=\frac{\mu_0icos\theta}{4\pi R}$$
	\item legge di Biot-Savart:
	      $$B=\frac{\mu_0 iu_\Phi}{2\pi R}$$
	\item fili paralleli distanti R:
	      $$F_{ab}=i_bL\times B_a=i_bL\frac{\mu_0iu_\Phi}{4\pi R}$$
	\item spira circolare:
	      $$B=\frac{\mu_0i}{2 R}u_n$$
	\item equazioni di Maxwell nel vuoto per un campo statico (j densità di corrente):
	      \begin{itemize}
		      \item $$\oint F\,dA=\frac{Q}{\varepsilon_0}$$
		      \item  $$\oint B\,dA=0$$
		      \item  $$\oint E\,ds=0$$
		      \item  $$\oint B\,ds=\mu_0 j$$
	      \end{itemize}
	      in caso di campo non statico le ultime due cambiano:
	      \begin{itemize}
		      \item  $$\oint E\,ds=\frac{-d\Phi_B}{dt}$$
		      \item  $$\oint B\,ds=\mu_0 j+\mu_0\varepsilon_0\frac{\partial E}{\partial t}$$
	      \end{itemize}
	\item particella in rotazione in un campo magnetico:
	      $$r=\frac{vm}{qB}$$
	      $$T=\frac{2\pi r}{v}=\frac{2\pi m}{qB}$$

\end{itemize}
\end{document}