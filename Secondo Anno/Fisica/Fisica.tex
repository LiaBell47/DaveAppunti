\documentclass[12pt, a4paper, openany, oneside]{book}
\usepackage[italian]{babel}
\usepackage[T1]{fontenc}
\usepackage[utf8]{inputenc}
\usepackage{amsmath} 
\usepackage{xcolor}
\usepackage[margin=1in]{geometry} 
%usepackage[latin1]{inputenc}
\begin{document}
\pagestyle{plain}
\author{DaveRhapsody}
\title{Fisica}
\date{30 Settembre 2019}
\maketitle
\tableofcontents
\chapter{Introduzione al corso}
Non è presente materiale didattico, le lezioni sono architettate in modo che
si segua dalla lavagna, è consigliato dal prof stesso di usare gli appunti od i
libri che (per coloro che han fatto fisica) si usavano alle superiori.
\newline \newline
Il programma è \textbf{tutta la fisica} in generale, ma affrontata in modo 
semplice, quasi banale, l'ultimo argomento dovrebbe essere il magnetismo, 
immaginatevi
quanto (non) si farà di quell'argomento. Ci sono 5 appelli in un anno, il primo
sarà a gennaio, poi febbraio, giugno, luglio e settembre, MA Gennaio e 
Febbraio dell'anno dopo sono inclusi
\newline \newline
Il che significa che io posso fare i due parziali e poi fare l'orale anche a
Febbraio. Noi possiamo iscriverci solo allo scritto, e verremo spostati 
all'orale SE siamo già sufficienti. 
\section*{Alcune osservazioni}
Lo studio della fisica nasce dall'osservazione di una serie di fenomeni che 
accadono, con lo scopo di misurarli ed infine dimostrare il perchè questi
si verificano, 
\newline \newline
Esistono una serie di \textbf{modelli} che sono in grado di descrivere ciò che
noi vediamo, ad esempio quando vedremo il moto, noi diremo "Osserviamo il moto
di un corpo", con corpo inteso come punto. Il punto è un oggetto di dimensioni
infinitesimali, e nel caso del moto ne analizzeremo i dettagli in modo 
specifico.
\newline \newline
La nostra teoria parte da un modello semplificato che consente di capire il 
funzionamento di ciò che abbiamo di fronte. Nel caso dei Gas ad esempio ci 
saranno arricchimenti dei modelli (del tipo non esistono solo gas perfetti)
etc. \newline \newline
Noi dobbiamo cercare di trovare il modello minimo, più semplice in grado di
\textbf{descrivere} una cosa. In fisica si adotta un atteggiamento \textbf{
Deduttivo}
, infatti non si ragiona generalmente in modo induttivo. Consideriamo che non 
esiste un modello finale che non si possa contradire.
\section{Cosa ci servirà}
Iniziamo definendo alcune quantità che ci interesseranno, ovvero massa, spazio 
e tempo.
\newline \newline
C'è bisogno di capire che quantità si stia misurando, quindi si usano le unità 
di misura che cosa oggettivamente stiamo quantificando. Immaginatevi cosa 
significhi quantificare senza unità di misura. (Per dire Galileo usava i 
battiti del cuore.)
\newline \newline
\subsection{Dal punto di vista numerico}
Si ha che in qualsiasi campo si ha un ordine di grandezza, ogni fenomeno ha la
propria scala da usare, ci saranno i coefficienti di riferimento, i prefissi
($micro mm, n$), c'è un vero e proprio intervallo di grandezze ($10^{n}$).
\section{Notazione scientifica}
Ecco un esempio di numero scritto in notazione scientifica: 
\[
5 x 10^{5} = 50000 \newline
8 x 10^{-1} = 0,1 
\]
\newline \newline
Ragionando su come sono composti, abbiamo le cifre significative, ovvero cifre
che hanno senso di essere tenute in considerazione. In che senso? Se devo 
misurare un banco di scuola posso dire che è tipo 1034 mm, OPPURE dire che è un
metro e 34 millimetri.. E' la stessa cosa, ok, detta in modi diversi
\newline \newline
Se specifico una cifra (tipo anche) lo 0 in un $0,12320$ esso è cifra significativa!
\newline \newline
Se ho invece un numero tipo $1,010$ posso scriverlo in due modi
\begin{itemize}
	\item 1,011 +/- (Ok non so come si fa il + e - in \LaTeX) 0,001
	\item 1,01 
\end{itemize}
Nulla di estremamente complesso ma va detto comunque, per dire se ho $1,234567$
posso approssimarlo in $1,23457$. \newline \newline
\paragraph{ATTENZIONE} Nel caso della \textbf{NOTAZIONE SCIENTIFICA} si tiene
in considerazione la parte numerica $\neq$ 0. tipo $123.000.000$ ha 3 cifre che 
sono proprio 123
\chapter{Cinematica}
E' la branca della fisica che si occupa di descrivere la traiettoria di un 
corpo, dovremo predirla, calcolarla, basandosi su un campo di forza, uno spazio
, introdurremo la forza in grado di cambiare il moto di un corpo MA per prima 
cosa
\section{Come definiamo la traiettoria di un corpo}
Definiamo la differenza tra grandezza scalare e vettoriale
\begin{itemize}
	\item Le grandezze vettoriali hanno con sè una direzione, un verso, ed un modulo definito anche intensità. L'esempio per eccellenza è lo spostamento e la velocità.
	\item Le grandezze scalari sono valori precisi fissi, dei valori che indicano qualcosa di quantitativo più che qualitativo.
\end{itemize}
\subsection{Esempio di grandezza vettoriale}
Supponiamo di avere due punti $x_{0} e x_{1}$ ponendoli distanti $\lambda$ tra 
loro. $\lambda$ sarà coincidente con $x_{0} - x_{1}$. Per definire il verso
basta osservare chi è il minimo tra $x_{0} e x_{1}$, lo si vede graficamente,
oppure osservando chi dei due è il maggiore. 
\newline \newline
Da un lato abbiamo un vettore (ancora monodimensionale), ma abbiamo anche 
dato un piano dimensionale, per esprimere il concetto di vettore relativo alla
posizione del nostro punto.
\newline \newline
Il sistema di riferimento è il sistema cartesiano, in questo caso Monoasse 
pertanto ci basta avere solo la $x$. $x_{0} e x_{1}$ sono semplicemente dei 
punti, ma hanno un nome specifico, in questo caso sono delle vere e proprie
posizioni.
\newline \newline
Come si diceva prima, per capire il \textbf{Verso} bisogna osservare la differenza
tra $x_{0} e x_{1}$, se negativa allora va all'indietro, al contrario andrebbe 
avanti molto semplicemente
\subsection{L'esempio di una palla che cade in un piano inclinato}
Il nostro punto materiale è la palla, e per capire lo spostamento bisogna
tracciare un grafico che indica le posizioni lungo le quali la pallina passa,
quindi si semplifica tutto con un grafico a singolo asse.
\newline \newline
Chiaro che se ho un modello \textbf{Dinamico} è un problemino diverso perchè
avrei anche forze tipo la grafità etc, ma per ora descriviamo questo moto.
\newline \newline
La pallina parte dalla posizione $p_{0}$ e passerà per un $p_{1,2,3,4}$ aventi
una serie di tempi passati dall'istante 0 che si chiameranno $t_{1,2,3,4}$ etc.
\newline
Per descrivere questo bisogna trovare una legge che sia in grado di esprimere
per qualsiasi istante quali possano essere le condizioni. 
$$
\begin{cases}
t_{\lambda} = $tempo richiesto per arrivare dalla posizione $p_{0}$ a
$p_{\lambda}  \\

\end{cases}
$$ 
\subsection{Alcune precisazioni}
\begin{itemize}
	\item Lo spostamento è la distanza in linea d'aria
	\item La distanza percorsa può essere nettamente maggiore
\end{itemize}
Lo spostamento è vettoriale, la distanza percorsa è uno scalare
\section{La velocità}
E' la quantità di spazio(s) percorsa da un corpo in un determinato tempo(t),
specificando che ci sia la distanza percorsa e lo spostamento.
\newline \newline
Attenzione, prima c'è da tenere conto della differenza dei tempi, che chiameremo 
$\Delta t = t_{Finale} - t_{Iniziale}$
\newline
Abbiamo 3 velocità:
\begin{itemize}
	\item {Velocità media scalare: $v_{media} = \frac{distanza~percorsa}{\Delta
	t}$ che è la distanza percorsa sul tempo passato da quando son partito a 
	quando sono arrivato }
	\item {Velocità media vettoriale: $\vec{v} = \frac{\vec{\Delta x}}
	{\Delta t}$ con $\Delta x$ che è il vettore spostamento tra la posizione 
	$p_{iniziale}$ e $p_{finale}$ }
\end{itemize}
\subsection*{Osservazione:}
Ragionando per formule inverse, se voglio capire quanto ho percorso mi basta 
fare $d = \Delta t * v_{media}$ , ma in realtà non è propriamente corretto. 
\newline \newline

\end{document}