\documentclass[12pt, a4paper, openany, oneside]{book}
\usepackage[italian]{babel}
\usepackage[T1]{fontenc}
\usepackage[utf8]{inputenc}
\usepackage{amsmath} 
\usepackage{xcolor}
%usepackage[latin1]{inputenc}
\begin{document}
\pagestyle{headings}
\author{DaveRhapsody}
\title{Linguaggi di Programmazione}
\color{blue}
\date{30 Settembre 2019}
\color{black}
\maketitle
\tableofcontents
\chapter{Introduzione al corso}
\section{Programma del corso}
Il corso è volto ad insegnare dei paradigmi di programmazione dei seguenti tipi:
\subsection{Logica Matematica e Linguaggi logici (Prolog)}
Termini, fatti(predicati), regole, unificazione, procedura di risoluzione
\subsection{Linguaggi funzionali e Lisp (et al.)}
Atomi, liste, funzioni e ricorsione
\subsection{Linguaggi imperativi} % (fold)
\label{sub:linguaggi_imperativi}
Memoria, stato, assegnamenti, puntatori
\newline \newline
{\color{black} \rule{\linewidth}{0.3mm} }
\newline
Il concetto è che con questo corso si vanno a studiare paradigmi più evoluti,
usati tutt'ora e comunque aventi un ampio approccio logico, oltretutto LISP è 
usato nelle pagine web (Si userà moltissimo la ricorsione, A I U T O)
\section{Modalità d'esame}
\label{sec:modalità_d_esame}
\subsection{subsection name}
\label{sub:subsection_name}
\begin{itemize}
	\item il voto finale sarà una media pesata dei voti conseguiti nell'esame relativo alla parte teorica e nell'esame del progetto 
	\begin{itemize}
		\item Occhio, il peso è a discrezione dei prof
	\end{itemize}
\end{itemize}
\subsection{Prove parziali}
\label{sub:prove_parziali}
Le prove d'esame sono costituite da uno scritto di 6-10 domande, e da un progetto da consegnare entro una data prefissata
\section{Appelli regolari}
\label{sec:appelli_regolari}
Gli appelli regolari sono composti da un progetto ed un esame scritto, che può
essere seguito da un esame orale a discrezione del docente basato sui temi trattati durante il corso
\newline
\textbf{NON C'E' POSSIBILITA' DI RECUPERI}, infatti scritto, orale e progetto vanno sostenuti \color{red} \textbf{NELLO STESSO APPELLO} \color{black}
\newline
Progetto e scritto sono corretti separatamente
\paragraph{NON CI SARANNO ECCEZIONI}
\label{par:non_ci_saranno_eccezioni}
Lo avete già letto nel passaggio precedente, ma lo ripeto lo stesso perchè 
deve essere chiaro che N O N  S I  F A N N O  E C C E Z I O N I.
%------------------------------------------------------------------------------
\chapter{Il paradigma}
\section{Cos'è?}
E' il metodo di soluzione ad undeterminato problema, a seconda dei paradigmi si
hanno diversi tipi di linguaggi di programmazione
\section{Storicamente}
Il primo paradigma è l'imperativo, cioè il paradigma basato sui tre costrutti
di selezione, iterazione e sequenza. \newline
Inoltre si mantiene il concetto di assegnamento di un valore ad una determinata variabile
\subsection{L'effetto collaterale}
Successivamente vedremo che in Prolog ci saranno parecchi problemi se provassimo
ad assegnare direttamente un valore ad una variabile
\section{Logica del primo ordine}
Prolog è costituito da una serie di clausole derivanti dalla logica del primo
ordine
\section{Linguaggi funzionali}
Questi si basano proprio sui concetti matematici di funzione, ad esempio si 
ragiona sui domini, sui codomini, sull'insiemistica, solite cose
%------------------------------------------------------------------------------
\chapter{Paradigma imperativo}
Le caratteristiche essenziali dei linguaggi imperativi sono legate all'archit-
tettura di Von Neumann, costituita dai famosi due componenti \textbf{Memoria (componente passiva) e Processore (componente attiva)}
\newline
In pratica la principale attività che ha la cpu è quella di eseguire calcoli
ed assegnare valori alle variabili, che sono delle celle di memoria.
\paragraph{Va considerato}
Il concetto di variabile è un'astrazione di una cella di memoria, per dire 
se giochi su assembly vai a toccare i veri e propri registri, mentre su C o 
Assembly si ragiona per nome di variabile, non vai di indirizzamento fisico
%------------------------------------------------------------------------------
\section{Il concetto di variabile}
In Prolog e LISP cambia completamente il concetto di variabile, ma per come 
saranno presentati vedremo che non c'entra niente. \newline
In matematica abbiamo il concetto di variabile? Sì, quella che sta dentro una
funzione, in informatica è diciamo diverso, non è un'astrazione, ma lo vedremo
in seguito
\section{Modello di Von Neumann}
Per manipolare la memoria utilizzo la variabile, simbolo che indica la cella 
di memoria, nei linguaggi funzionali sarà possibile usare il concetto di 
variabile matematica. 
\newline \newline
Alla fine il modello di Neumann è composto da I/O, Memoria e CPU con i suoi 
cicli di clock
\section{Stile prescrittivo}
Un programma scritto in un linguaggio imperativo prescrive le operazioni che
la CPU deve eseguire per modificare lo stato di un sistema \newline \newline
Le istruzioni sono eseguite nell'ordine in cui queste appaiono, ad eccezione
delle strutture di controllo
\paragraph{Realizzati} sia attraverso interpretazione che compilazione, nati
più per manipolazione numerica che simbolica.
\newline \newline
\section{Concetto di programma}
Un programma è intendibile come un insieme di algoritmi e di strutture dati
ma la struttura di un programma consiste in
\begin{itemize}
	\item Una parte dichiarativa in cui son presenti le dichiarazioni di tutte
	le variabili del programma e del loro tipo
	\item Una parte che descrive l'algoritmo risolutivo utilizzato, mediante istruzioni del linguaggio	
\end{itemize}
\section{Perchè utilizzare paradigmi diversi?}
Per esempio l'intelligenza artificiale si sviluppa su linguaggi di 
programmazione specifici, bisogna usare linguaggi che operino in un 
determinato modo, considerati tipo di Altissimo super mega galatticissifantastico livello infatti, utilizzabili pure da non programatori
\newline \newline
Infatti son generati per manipolazione simbolica non numerica 
\section{Paradigma logico}
Concetto primitivo: Deduzione logica, avente una base di logica formale e un 
obbiettivo, che è intendibile come formalizzazione del ragionamento
\newline \newline
\textbf{Programmare infatti significa} descrivere il problema con frasi 
(Formule logiche) del linguaggio, \newline
Interrogare il sistema, che effettua deduzioni in base alla "conoscenza 
rappresentata" 
\paragraph{Ai lettori} Mi rendo conto che non si capisca un cazzo, voi 
immaginatevi come mi stia sentendo al momento io mentre prendo appunti.. 
Perdonatemi
\newline \newline
Prolog è un insieme di formule ben formate, ragiona con il linguaggio logico, 
con una descrizione della realtà di interesse, di fatto è una dimostrazione in
un linguaggio logico che costituiscono un programma, poi ho una frase da dare
al mio interprete, Prolog icchè fa? Semplicemente la realizza sotto forma di 
dimostrazione. 
\newline \newline
\section{Esempio di un programma Prolog}
Ci sono fondamentalmente:
\begin{itemize}
	\item Asserzioni incondizionate (\color{red} fatti \color{black})
	A.
	\item Asserzioni condizionate (\color{red} regole \color{black})
	A :- B, C, D, ... , Z.
	\begin{itemize}
		\item A è la conclusione o conseguente (deve avere una sola clausola)
		\item B, C, D, \dots,Z sono le premesse o antecedenti 
	\end{itemize} 
	\item Un'\color{red}interrogazione \color{black} ha la forma: :- K, L, M, 
	\dots, P. \newline
\end{itemize}
Ovviamente A, B, C, *TUTTE LE ALTRE*, sono semplicemente predicati
\newline
MI RACCOMANDO MASSIMA ATTENZIONE ALLA SINTASSI, ogni clausola Prolog termina
con un punto. \newline \newline
La ',' si legge come AND
\subsection{Esempio:}
Due individui sono colleghi se lavorano per la stessa ditta/azienda
\color{red} Regole \color{blue} Fatti \color{black} Interrogazione
\newline \newline
\color{red}
collega(X, Y) :-  \newline	
lavora(X, Z), \newline	
lavora(Y, Z), \newline	
diverso(X, Y).
\newline \newline
\color{blue}
lavora(ciro, ibm)  \newline
lavora(ugo, ibm)  \newline
lavora(olivia, samsung)  \newline
lavora(ernesto, olivetti)  \newline
lavora(enrica, samsung)
\newline \newline
\color{black}
:- collega(X, Y). 


\end{document}